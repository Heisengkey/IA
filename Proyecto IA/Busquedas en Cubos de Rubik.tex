\documentclass[11pt]{article}
\usepackage{fullpage}
\title{Comparativa de distintas representaciones para busquedas Cubo Rubik}
\author{Pedro Viegas Peñalosa}
\date{}

\begin{document}
\maketitle
\section{Conceptos Previos}
	Sabemos que un cubo de Rubik es un puzzle, en el que tenemos bloques que giran sobre un el centro de una cara, pudiendo estas piezas rotarse segun la secuencia de movimientos realizada.

	Para un cubo de $2\times 2$ tendremos 8 piezas, teniendo cada una 3 posibles rotaciones. Esto nos deja con $8\,! \times 3^{8} = 264.539.520$ posiciones.
	En primer lugar vamos a desglosar los posibles movimientos que podemos realizar, con objetivo de poder componer un buen espacio de estados y sus posibles transicciones.

\subsection{Notacion}
	Para nombrar las caras del cubo, se usa una notación de tal forma que usamos la inicial del nombre en inglés de la cara. Estas serían:
\begin{itemize}
	\item{$Up \rightarrow U$}
	\item{$Down \rightarrow D$}
	\item{$Right \rightarrow R$}
	\item{$Left \rightarrow L$}
	\item{$Front \rightarrow F$}
	\item{$Back \rightarrow B$}
\end{itemize}
	Los posibles movimientos los he catalogado en:

\begin{itemize}
	\item Elementales
	\item Rotaciones
	\item Secuencias
\end{itemize}

	Iremos explicando cada uno de estos, y como los tendremos en cuenta en nuestra busqueda según como estos funcionan.

\section{Movimientos}

\subsection{Elementales}
	Son los giros de cada cara, individualmente. Se les identifica a cada uno, por la inicial del nombre de la cara, usando la notación.
	Estos giros son siempre en el sentido horario correspondiente si hacemos la traslación de esa cara a la $Front$. Los giros antihorarios se denotan como la inicial correspondiente a la cara, seguida de $'$, leyendose como prima.
	
	Como se puede deducir, los movimientos elementales primos, se pueden representar como 3 giros consecutivos horarios. Por este mismo motivo podríamos no tenerlos en cuenta en la generación de los estados hijos del actual, reduciendo el número de posibles transiciones, pero teniendo en cuenta estos movimientos al final del algoritmo, postprocesando el resultado de la busqueda y convirtiendo 3 movimientos consecutivos de la misma cara como un movimiento primo.
\subsection{Rotaciones}
	Por rotaciones denotamos a los giros completos del cubo, ya sea por su rotación del eje $x, y $ó$ z$, y estos giros se denotaran con esa misma notación, siendo en sentido horario y existiendo las rotaciones primas. 

Teniendo en cuenta como funcionan las rotaciones, podemos darnos cuenta que realmente una rotación podría ser considerada como el giro de las 3 capas que conforman 1 cara. (representar con una imagen para más claridad)

Por este mismo motivo, vamos a  omitir tambien los giros de las posibles transiciones, con fin de evitar la gran cantidad de posibles transiciones en una busqueda BFS. Si bien, para A* si podriamos tenerlas en cuenta.
Trataremos las rotacions como los movimientos Elementales primos, postprocesando el resultado del algoritmo y mostrando el numero de movimientos que se hace en realidad, ya que las rotaciones no cuentan como un movimiento realmente. (en verdad si, pero explicar mejor si hace falta)

\subsection{Secuencias}


\end{document}