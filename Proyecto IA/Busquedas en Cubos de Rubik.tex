\documentclass[11pt]{article}
\usepackage{fullpage}
\title{Comparativa de distintas representaciones para busquedas Cubo Rubik}
\author{Pedro Viegas Peñalosa}
\date{}

\begin{document}
\maketitle
\section{Conceptos Previos}
	Sabemos que un cubo de Rubik es un puzzle, en el que tenemos bloques que giran sobre un el centro de una cara, pudiendo estas piezas rotarse segun la secuencia de movimientos realizada.

	Para un cubo de $2\times 2$ tendremos 8 piezas, teniendo cada una 3 posibles rotaciones. Esto nos deja con $8\,! \times 3^{8} = 264.539.520$ posiciones.
	En primer lugar vamos a desglosar los posibles movimientos que podemos realizar, con objetivo de poder componer un buen espacio de estados y sus posibles transicciones.

	Los posibles movimientos los he catalogado en:

\begin{itemize}
	\item Elementales
	\item Rotaciones
	\item Secuencias
\end{itemize}

	Iremos explicando cada uno de estos, y como los tendremos en cuenta en nuestra busqueda según como estos funcionan.

\section{Movimientos}

\subsection{Elementales}
\subsection{Rotaciones}
\subsection{Secuencias}


\end{document}